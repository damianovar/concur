
% Step 2 - revise the preliminary ''per course'' Knowledge Components Lists
% [expected completion time: 15 minutes]
% Open the ''program-wide knowledge components list'' file, and go to the column that corresponds to the course you are supposed to focus on
% Take the syllabus and the textbook of your course
% By reading this syllabus and browsing the textbook, identify the 20 most important KCs that identify that course, and write them down in the column of the KCs list file
% when you are at this point, inform the organizers that you reached this step (this is important to do an ''in-worskhop'' tuning of how much time shall be spent in the various steps)
% Step 3 - merge the preliminary ''per course'' Knowledge Components Lists into a ''program-wide'' list
% [expected completion time: 60 minutes]
% start from the first course: now the teacher associated to that course will be the ''presente''r
% the presenter starts reading the first KC that is present in her/his list. Then the whole group starts a discussion and decides, collaboratively:
% if that is actually a KC
% if it is a relevant KC
% if that name is a good one
% if it should be splitted in more KCs
% if it should be merged with other KCs
% if eventually the group decided that that KC is a ''good one'', then the presenter moves that KC into the ''Main Knowledge Components'' column. Potentially, she/he adds some notes in the relative column
% in the while, all the other teachers individually update their ''per-course'' lists, eliminating duplicates (so that they won't need to be rediscussed later on)
% -- important: do NOT start discussing ''who teaches this'' now: we will do it later on, so let's focus on completing this step now --
% then the presenter reads the second KC in her/his list, and we repeat the procedure above up to finish the whole ''per-course'' list
% then we continue with the second course: now the teacher of this course becomes the presenter, and we repeat the same ''per-KC'' discussion as above
% when we finish all the courses we are done with the first part of the workshop.
% 
% Good job!
% 
% How to fill up the Knowledge Components Matrix form
% Step 1 - get acquainted with the following terminology
% [expected completion time: 3 minutes]
% Step 2 - get acquainted with the following color scheme
% [expected completion time: 3 minutes]
% gray: parts that are comments or indications (i.e., cells that you should not modify);
% pink: parts that you have to fill (i.e., if you don't fill them then you won't get results); 
% yellow: parts that you should fill (i.e., if you don't fill them then you will get only partial results, and not everything that the tool may give you); 
% green: parts that you should fill if you want to get all the results that the suite can compute.
% Thus: we aim at filling up at least all the pink parts. We will do the yellow and green ones only if we will have time (and will).
% Step 3 - fill up the pink parts of the ''Knowledge components matrix for''m
% [expected completion time: 45 minutes]
% BE SURE TO KNOW ON WHICH COURSE YOU ARE SUPPOSED TO WORK (and that you will compile the right ''Knowledge Components Matrix'' file). If you don't know on which course you should work, discuss this with your colleagues
% open the ''program-wide Knowledge Components list'' file in the google-drive folder in one window
% select from the ''program-wide KCs list'' the 10-15 most important KCs characterizing the prerequisites of your course, and the 10-15 most important KCs characterizing what your course develops. For your convenience, there is some allocated space from row 30 and row 50 of the ''program-wide KCs list'' file where you can write down these things
% open the specific ''Knowledge Components Matrix'' file that is dedicated to your specific course in the google-drive folder in an other window. Be sure you open the right one! Once you opened it, go to its ''course summary'' tab
% copy the prerequisite and developed KCs selected in sub-step 3 above, and paste (with the ''values only'' option!) them in the ''course summary'' tab of your specific ''KC Matrix'' file. Note: this tab works as a ''database'', in the sense that by filling up this tab you will automatically fill up a lot of grey cells in the remaining tabs
% [this step is essential] fill up the course starting / course ending dates in this ''course summary'' tab. As for the year, pretend that we start from 2020. So if it is a ''second-year course'', then put 2022; fourth year, 2024; and so on 
% fill up the ''taxonomy types'' field in this ''course summary'' tab - if you are unsure about what to write, ask the organizers of the workshop
% when you are at this point, inform the organizers that you reached this step (this is important to do an ''in-worskhop'' tuning of how much time shall be spent in the various steps. Note also that from now on you will work only on your specific ''KC Matrix'' file)
% go to the ''developed vs prerequisite KCs'' tab, and fill up each ''input-output'' couple in this way:
% go to a certain row, that will correspond to a certain developed KC, say X; 
% go to a certain column, that will correspond to a certain either prerequisite or developed KC, say Y; 
% fill that cell with which minimum taxonomical knowledge level the student should have reached about Y to be able to learn X in a satisfactory way. Note that if you did not fill up all 20 slots for the CKs in the course summary tab then some rows and/or columns will have a ''0'' in the CP. That rows and/or columns should not be filled, obviously.
% go to the ''info on the developed KCs'' tab, and fill up the column indicating which taxonomical knowledge level students should reach about the KCs that are developed by that course.
% Step 4 - fill up the yellow parts of the ''Knowledge Components Matrix for''m
% [expected completion time: 20 minutes]
% be sure to have filled up all the pink parts; 
% go to the ''course summary'' tab, and fill up the list of TLAs included in your course;
% go to the ''developed KCs vs TLAs'' tab, and fill up a matrix in a way that is similar to how it was done in the ''developed vs prerequisite KCs tab''. I.e., fill up each ''input-output'' couple in this way:
% go to a certain row, that will correspond to a certain TLA, say X;
% go to a certain column, that will correspond to a certain either prerequisite or developed KC, say Y;
% fill that cell with which taxonomical knowledge level the student should have reached about Y to be able to execute or follow X in a satisfactory way. Note that it may be that the TLA either depends on that KC or helps developing that KC. For now this form does not capture this information; thus by inserting the information you just indicate that that TLA is in a sense connected to that KC.
% Step 5 - fill up the green parts of the ''Knowledge Components Matrix for''m
% [expected completion time: 30 minutes]
% The steps are actually similar to what you did up to now, so there is no need to repeat things. Go through all 







% The cells in the excel file have 4 different background colors, with these meanings:
% ? gray: parts that are comments or indications (i.e., cells that you should not modify); 
% ? pink: parts that you have to fill (i.e., if you don't fill them then you won't get results); 
% ? yellow: parts that you should fill (i.e., if you don't fill them then you will get only partial results, and not everything that the tool may give you); 
% ? green: parts that you should fill if you want to get all the results that the suite can compute. 
% Thus: if you see pink, then fill it up. If you see yellow or green, do it if you have time and will. The summary of what you get by doing different levels of filling up is in Section 5.
