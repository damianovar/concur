\section{Analysing the data} \label{sec:analysing_the_data}

\subsection{Launching Matlab}

To open the COnCUR app, Matlab 2019a or later is required. The KCM analysis
part, instead, requires Matlab 2018a or a higher version.

Once Matlab is launched, navigate to the ``Scripts'' folder in Matlab (cf.\
Figure~\ref{fig:file-structure-base-folder}). Once there, execute
\texttt{main.m}.

\subsection{The initial prompt}

Launching \texttt{main.m} and waiting a few moments opens a prompt like the
one in Listing~\ref{lst:main-menu}.

\begin{lstlisting}[caption=The main menu.,label=lst:main-menu]
Main Menu: what would you like to do? Actions available:
1 - select a full program to analyze (currently loaded: LTU-MachineElements-2019)
2 - plot the selected program (LTU-MachineElements-2019)
3 - open the COnCUR app
4 - select which KCM you want to analyze
5 - show the currently selected KCM  (now: M0009T, M0012T, M0013T, M7007T, T0015T)
6 - analyze the currently selected KCM  (now: M0009T, M0012T, M0013T, M7007T, T0015T)
7 - generate a report of the currently selected program (LTU-MachineElements-2019)
8 - list the editable parameters
9 - change the parameters
10 - exit COnCUR
your choice: 
\end{lstlisting}

The most interesting (and not self-explaining) items are:
%
\begin{itemize}
\item ``plot the selected program'', that will produce a figure that
	summarizes which course connects with which other in terms of
	\acp{KC}. Note that the suite will assign all the
	prerequisite \acp{KC} that are not taught by any course to an
	artificial ``prerequisites'' course. Note also that the plot is
	explorable - in the sense that clicking on the various arrows will
	expand which \acp{KC} form the connections; 
\item ``open the COnCUR app'' - for this item see
	Section~\ref{ssec:the_cite_app};
\item ``analyse the currently selected KCM'' - for this item see
	Section~\ref{ssec:the_kcm_analysis_prompt}.
\end{itemize}


\subsection{The KCM analysis prompt}
\label{ssec:the_kcm_analysis_prompt}

Selecting to analyze the currently selected KCM will produce a prompt like
the one in Listing~\ref{lst:KCM-analysis-menu}.

\begin{lstlisting}[caption=Menu that one obtains by launching an analysis of the currently selected KCM.,label=lst:KCM-analysis-menu]
KCM Analysis Menu: what would you like to do? Actions available:
1 - list the available centrality indices
2 - select centrality index  (now: betweenness)
3 - plot the centrality indexes
4 - list the most central prerequisite KCs
5 - list the most central developed KCs
6 - exit the KCM analysis tool
your choice: 
\end{lstlisting}

The most interesting (and not self-explaining) items is ``plot the
centrality indexes'', which produces a figure that should be interpreted as
suggested in Section~\ref{sec:interpreting_the_results}.

\subsection{The COnCUR app}
\label{ssec:the_cite_app}

Once the app is launched, you are presented a blank graph plot with a list of
courses on the right. Select the course you would like to visualize by
clicking on it. Use shift-click to select a range of courses and
control-click (command-click on Mac OS) to select a set of courses. Pressing
the ``Plot'' button generates an explorable plot using the \acp{KCM} of the
selected courses. Clicking on a node highlights its connections. The colors
of the connections represent different taxonomical levels. Checking the
``Show Structural Problems Only'' box before plotting will hide everything
but structural issues in the \ac{KCG}, that may be either of a temporal
structure (i.e., a course assumes that a prerequisite \ac{KC} has been
taught, but it has actually been taught at a later course) or taxonomical
level structure (i.e., one assumes that a prerequisite has been previously
taught at a certain level, but actually it has been taught at a lower level).
The types of structural issues to reveal can be selected in the same way as
the courses to include in the plot.
