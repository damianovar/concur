\section{An overview of the structure of this manual}
\label{sec:an_overview_of_the_structure_of_this_manual}

The first part of the manual aims to explain how to use the software.

\begin{description}

\item[Section~\ref{sec:what_is_this_and_why_should_i_use_it}] describes the
	general purpose of the software, and frames the bigger picture;

\item[Section~\ref{sec:inserting_the_data}] describes how to collect and
	insert the data that is processed by the software;

\item[Section~\ref{sec:analysing_the_data}] describes how to use the
	software to produce information;

\item[Section~\ref{sec:interpreting_the_results}] describes how to interpret
	that information.

\end{description}


The appendix explains what the various things mean. % Should be reworded to better reflect the meaning.

\begin{description}

\item[Section~\ref{sec:lexicon}] is a glossary of the terms that are used
	throughout the manual; 

\item[Section~\ref{sec:how_to_define_the_knowledge_components_list}] is not
	yet implemented. It should describe how to identify and define the
	\acp{KC}. This section is ancillary to
	Section~\ref{sec:inserting_the_data};

\item[Section~\ref{sec:how_to_interpret_and_assign_taxonomy_levels}] is not
	yet implemented. It should describe how to interpret (and thus
	assign) knowledge taxonomy levels. Also this section is
	ancillary to Section~\ref{sec:inserting_the_data};

\item[Section~\ref{sec:how_to_identify_ilo}] is not yet implemented. It
	should describe how to identify and define the \acp{ILO}. Also this
	section is ancillary to
	Section~\ref{sec:inserting_the_data};

\item[Section~\ref{sec:how_to_identify_tla}] is not yet implemented. It
	should have the same purpose has the above item, and is dedicated to
	defining the \acp{TLA}. Also this section is ancillary to
	Section~\ref{sec:inserting_the_data};

\item[Section~\ref{sec:debugging}] collects a series of known bugs /
	non-ideal features in the software that may make Matlab stop
	running. Please consider that this software is more in
	alpha-testing than in beta-testing.

\end{description}

