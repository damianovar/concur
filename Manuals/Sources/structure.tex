\section{An overview of the structure of this manual}
\label{sec:an_overview_of_the_structure_of_this_manual}

The first part of the manual is dedicated more to "how to use the software":

\begin{description}

\item[Section~\ref{sec:what_is_this_and_why_should_i_use_it}] describes the general purpose of the software, and frames the bigger picture;

\item[Section~\ref{sec:inserting_the_data}] describes how to collect and insert the data that will be processed by the software;

\item[Section~\ref{sec:analysing_the_data}] describes how to use the software to produce some information;

\item[Section~\ref{sec:interpreting_the_results}] describes how to interpret that information.

\end{description}


The appendix is more about "what do the various things mean":

\begin{description}

\item[Section~\ref{sec:lexicon}] is a reference of meanings of the various terms that are used throughout the document; 

\item[Section~\ref{sec:how_to_define_the_knowledge_components_list}] describes how to identify and define the \acp{KC}. This section is ancillary to Section~\ref{sec:inserting_the_data};

\item[Section~\ref{sec:how_to_interpret_and_assign_taxonomy_levels}] describes how to interpret (and thus assign) knowledge taxonomy levels. Also this section is ancillary to Section~\ref{sec:inserting_the_data};

\item[Section~\ref{sec:how_to_identify_ilo}] describes how to identify and define the \acp{ILO}. Also this section is ancillary to Section~\ref{sec:inserting_the_data};

\item[Section~\ref{sec:how_to_identify_tla}] has the same purpose above, and is dedicated to defining the \acp{TLA}. Also this section is ancillary to Section~\ref{sec:inserting_the_data};

\item[Section~\ref{sec:debugging}] collects a series of known bugs / non-ideal features in the software that may make Matlab stop running. Please consider that this software is more in alpha-testing than in beta-testing.

\end{description}

