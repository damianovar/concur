\section{Interpreting the results}
\label{sec:interpreting_the_results}

\subsection{Interpreting the centrality indexes}

\begin{description}

	\item[out degree:] when considering a course, a high out-degree
		indicates that this course exposes students to \acp{KC} that
		are likely being introductory or instrumental to build a
		framework later on. This means that assessments in courses
		that have high out degrees correspond to "baseline
		assessments", i.e., evaluations of students' initial
		knowledge. Similar concepts apply to a \ac{KC}: a high out
		degree would imply that this \acp{KC} likely represents
		baseline knowledge;

	\item[in degree:] when considering a course, a high in-degree
		indicates that this course is the destination of the
		learning efforts. From a pedagogical perspective this
		indirectly indicates courses where it is important to engage
		students with active and higher-level learning activities.
		Assessments on these courses are a sort of "final
		assessments" for the program. Similar concepts apply to a
		\ac{KC};

	\item[betweenness:] when considering a course, a high betweenness
		indicates that this course serves as a key link between
		different clusters of courses or \acp{KC} in a program. This
		indicates that they serve as a bridge; for this reason
		assessments on courses with high betweenness serve both the
		purposes described above, i.e., as evaluations of students'
		initial knowledge for the next "cluster" of courses, and of
		students final knowledge for the previous "cluster" of
		courses. Similar concepts apply to a \ac{KC};

	\item[eigenvector centrality:] in graph theory, the eigenvector
		centrality measures the influence of a node in a network in
		a similar way that the Google pagerank algorithm works: if a
		node is pointed to by many nodes which also have high
		eigenvector centrality then that node will have high
		eigenvector centrality. In a sense, the links of a node with
		higher eigenvector centrality are more influential than the
		links of a node with a lower centrality index. When
		considering a course, an interpretation of this index is in
		terms of "how much this course serves as a support and
		reinforcement for other courses". Assessments on courses
		with high eigenvector centrality indexes serve as
		complementary assessments for other courses;

% 	\item[clustering coefficient:] Vectors with a high clustering coefficient are the ones most likely to efficiently enact change to the adjacent vectors. The clustering coefficient is useful in finding the course that can most efficiently implement change into adjacent courses. This is relevant because all assessment schemes require a feedback mechanism to modify the curriculum when the assessment data indicate a problem. The quicker this change can be incorporated into the curriculum, the sooner improved results will (hopefully) occur. Those courses with the highest clustering coefficient are the ones best suited for quick, efficient implementation.

	\item[incloseness:] yet to be interpreted
		% Something about being close to the predecessors of the
		% \ac{KC}.

	\item[outcloseness:] yet to be interpreted
		% Like incloseness, but for successors.

	\item[authorities:] yet to be interpreted

	\item[hubs:] yet to be interpreted

\end{description}


% \subsection{The connectivity indexes graph}
% 
% \emph{\ldots to be done \ldots}


% \subsection{Interpreting the \ac{KCG}}



