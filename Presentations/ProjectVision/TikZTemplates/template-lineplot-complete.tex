% % externalization of tikz figures									%
% \newif\ifexternalize
% % \externalizetrue
% \newcommand{\figurefilename}[1]{\ifexternalize \tikzsetnextfilename{#1} \fi}
% \ifexternalize
% 	\usetikzlibrary{external}
% 	\tikzexternalize[shell escape=-enable-write18]
% 	\tikzset{external/force remake}
% \fi
\figurefilename{tikz-courses-centrality-indexes-UU}

% for beamer animations
\tikzset
{
    invisible/.style={opacity=0},
    visible on/.style={alt={#1{}{invisible}}},
    alt/.code args={<#1>#2#3}{%
      \alt<#1>{\pgfkeysalso{#2}}{\pgfkeysalso{#3}} % \pgfkeysalso doesn't change the path
    },
}

\pgfplotscreateplotcyclelist{mycyclelist}{%
solid,	line width=0.02cm,	black,			opacity=0.8,	mark=none\\%
dashed,	line width=0.02cm,	black,			opacity=0.8,	mark=none\\%
solid,	line width=0.05cm,	black!30!white,	opacity=0.8,	mark=none\\%
dashed,	line width=0.05cm,	black!30!white,	opacity=0.8,	mark=none\\%
solid,	line width=0.07cm,	black!50!white,	opacity=0.8,	mark=none\\%
dashed,	line width=0.07cm,	black!50!white,	opacity=0.8,	mark=none\\%
}

\tikzstyle{every pin} = [fill = \SoftPrimary, draw = black, font = \footnotesize, align = center]

\pgfplotsset
{
	MyStyle/.style =
	{
		width = 0.7\columnwidth,
	}
}

\begin{tikzpicture}
[
	xscale	= 1,	% to scale horizontally everything but the text
	yscale	= 1,	% to scale vertically everything but the text
%	scale	= 0.6, every node/.style={transform shape}, % to scale everything, also the text
	remember picture, overlay	% to do absolute positioning
]
	%
	\begin{axis}
	[
		xmode					= log,			% can't go in user-defined style (limitations of pgfplots)
		ymode					= linear,		% can't go in user-defined style (limitations of pgfplots)
%		enlarge x limits		= 0.5,
		width					= 0.8\columnwidth,
		height					= 0.8\columnwidth,
		%
		axis x line				= bottom,	% bottom | middle | top
		every inner x axis line/.append style	= {-latex},
		hide x axis,
		x dir					= reverse,
		%
		axis y line				= left,		% left | middle | right
		every inner y axis line/.append style	= {-latex},
		hide y axis,
		y dir					= reverse,
		%
		axis on top,
		%
		xmin					= 0,
		xmax					= 1,
		ymin					= 0,
		ymax					= 1,
		%
		xlabel					= {$x$},
		xlabel style			= {align = right, text width = 1cm},
		xlabel near ticks,
		xlabel shift			= -15,
		every axis x label/.style		= {at={(xticklabel cs:1)}, anchor = west, yshift = 0.55cm, align = right, text width = 1cm},
		%
		xtick					= {0, 0.2, ..., 1},
		x tick style 			= {thick},
		xticklabels				= {0, 0.2, ..., 1},
		xticklabels from table	= {../Data/LTU.courses-centrality-indexes.tex}{xticks}
		x tick label style		= {rotate = -60, anchor = west},
		xmajorgrids,
		%
		extra x ticks			= {3, 7, 10},
		extra x tick labels		= {$t - N$, $t - T$, $t$},
		extra x tick style		= {grid = major, rotate = -60, anchor = west},
		%
		ylabel					= {$y$},
		ylabel near ticks,
		ylabel shift			= -15,
		y label style			= {rotate = -60, anchor = west},
		%
		ytick					= {0, 0.2, ..., 1},
		y tick style 			= {thick},
		yticklabels				= {0, 0.2, ..., 1},
		y tick label style		= {rotate = -60, anchor = west},
		ymajorgrids,
		%
		scaled x ticks			= base 10:-3, % will put 10^3 | notice that the sign is inverted
		scaled x ticks			= false,						% | will remove ticks with scientific notation
		y tick label style		= {/pgf/number format/fixed},	% |
		%
		every y tick scale label/.style = {at={(-0.07,1.1)}},	% to position the 10^x wherever one wants
		%
		cycle list name			= mycyclelist,
		%
		legend pos				= south east, % south west | north east | north west | outer south east | outer south west | outer north east | outer north west | 
		legend plot pos			= left, % right | none    [where to draw the line specifications]
		legend columns			= -1, % [-1 to draw all entries horizontally]
		legend entries			= {$\delta = 0.1 \;\;\;$, $\delta = 1$},
		legend style			=
		{
			draw				= none,
			fill				= none,
			only marks,
			inner xsep			= 0.1cm,
			inner ysep			= 0.1cm,
			nodes				= {inner ysep = 0.4cm, text depth = 0.15cm, text width = 1.5cm},
			cells				= {anchor = east},
% 			at					= {(1,0.5)}, % overrides 'legend pos'
% 			anchor				= west,
		},
		legend cell align		= left, % left | center | right
		%
		title style				= { text width = 6cm, align = center, font = \bf, yshift = 1cm },
		legend to name			= mylegend,
		reverse legend			= false, % true | false
		transpose legend		= false, % true | false
		%
		title					= {$p = b = 0.1$ \\ $N_{\max} = 200$},
		%
		axis background/.style	= {fill = black!05!white},
		grid style 				= {thin, dashed, black!20},
		%
		name					= firstplot,
		at						= {($(previousplot.east)+(1cm,0)$)},
% 		at						= (current page.center),
		anchor					= west,
		%
		% for plots having dates as coordinates
		% note: this requires including
		% \usepgfplotslibrary{dateplot}
% 		date coordinates in		= x,
% 		xticklabel				= {\hour:\minute},
% 		x tick label style		= {align = center},
% 		date ZERO				= {2013-06-25},
% 		xmin					= {2013-06-25 00:00},
% 		xmax					= {2013-06-29 23:59},
% 		xtick					= {2013-06-28 08:00,
% 								   2013-06-28 08:15,
% 								   2013-06-28 08:30,
% 								   2013-06-28 08:45,
% 								   2013-06-28 09:00}
	]
		%
		\foreach \e in {15,20,...,50}
		{
			% --------------------------------------------------------------
			\addplot
			[
				const plot, % smooth | sharp plot | const plot | const plot mark right | jump mark left | jump mark right | xbar | ybar | xbar interval | ybar interval | xcomb | ycomb | only marks
				dashed, % solid | dotted | densely dotted | loosely dotted | dashed | densely dashed | loosely dashed | dashdotted | densely dashdotted | loosely dashdotted | dashdotdotted | densely dashdotdotted | loosely dashdotdotted 
				draw			= black,
				mark			= none, % triangle | triangle* | diamond | diamond* | square | square* | pentagon | pentagon* | none
				line width		= 0.06cm,
				area legend, % line legend | empty legend | xbar legend | xbar interval legend | mesh legend
				visible on		= <1->, % for beamer animations
% 				forget plot, % if one does not want this plot to appear in the legend
			]
			table
			[
% 				col sep			= comma, % for plots having dates as coordinates
% 				trim cells		= true,
				x				= actual,
				y				= bound3sigma,
			]
			{../DatavisualizationsTables/file_\e.txt};
			\addlegendentry{$\delta = 0.1$};
% 			\addlegendentryexpanded{$\delta = \e$}; % in case one needs to expand the argument of the foreach
% 			coordinates
% 			{
% 				(0,0)
% 				(1,2)
% 				(2,8)
% 			};
% 			\closedcycle; % NOTE: in case you use this then you have to remove the ';' in the row above!
		}
		%
		\node[coordinate, pin={[pin distance = 2cm]above:{$\alpha = 0.08, N = 5573$}}] at (axis cs:4.17,2673.92) {};
		%
		% --------------------------------------------------------------
		\addplot
		[
			mesh,
			draw 	= black,
			opacity	= 0.5,
			domain	= 0:1,
			dashed,
		]
		{x};
		%
		% --------------------------------------------------------------
		\node at (axis cs:  0.3, 1)	{$\mathcal{L}^+$};
		%
	\end{axis}
	%
\end{tikzpicture}



