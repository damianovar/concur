% \textbf{Title: Fields 7}

How does the Skin effect affect AC-currents in conductors?\\

a. The current density is even in the cross section.

*b. The current is higher in the surface than in the core.

c. The resistance in the conductor decreases.

d. The current is concentrated to the core of the conductor.

e. I do not know.\\
