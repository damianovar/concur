% \textbf{Title: Complex numbers 8}

If we have a complex number z and want to take the \(n\)-th root of
it, how can we express \(\sqrt[n]{z}\) in polar coordinates?\\

a.
\(\sqrt[n]{z} = (\cos(n(\theta + 2k\pi))\ +\sin(n(\theta + 2k\pi))i)\),
where \(\theta\) is the phase and \(r\) is the magnitude of \(z\). Also \(k\) is an
integer such as \(0 \leq k \leq n - 1\).

b.
\(\sqrt[n]{z} = \sqrt[n]{r}(\cos(n(\theta + 2k\pi)) + \sin(n(\theta + 2k\pi))i)\),
where \(\theta\) is the phase and \(r\) is the magnitude of \(z\). Also \(k\) is an
integer such as \(0 \leq k \leq n - 1\).

*c.
\(\sqrt[n]{z} = \sqrt[n]{r}(\cos(\frac{(\theta + 2k\pi)}{n})\  + \sin(\frac{(\theta + 2k\pi)}{n})i)\),
where \(\theta\) is the phase and \(r\) is the magnitude of \(z\). Also \(k\) is an
integer such as \(0 \leq k \leq n - 1\).

d.
\(\sqrt[n]{z} = (\cos(\frac{(\theta  + 2k\pi)}{n}) + \sin(\frac{(\theta + 2k\pi)}{n})i)\),
where \(\theta\) is the phase and \(r\) is the magnitude of \(z\). Also \(k\) is an
integer such as \(0 \leq k \leq n - 1\).

e. I do not know.\\
